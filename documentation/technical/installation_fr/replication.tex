\chapter{Mettre en place une réplication de la base postgresql vers un autre serveur}

\section{Présentation}
L'objectif de ce chapitre est de présenter comment mettre en œuvre une réplication entre deux serveurs Postgresql, pour éviter toute perte accidentelle d'un enregistrement.

Il a été écrit par Alexandra Darrieutort, stagiaire à Irstea en 2016, et complété par Jacques Foury, responsable informatique du centre Irstea de Cestas (33), qui se sont inspirés de divers documents \cite{digitOcean} \cite{zeroPostgres} \cite{replicationPostgres} \cite{replicationTutorial}.

\subsection{Besoins exprimés}

Mise en place d'une réplication d'un serveur postgreSQL de sorte qu'il y ait préservation des données, c'est-à-dire qu'une écriture faite sur le serveur maître se retrouve sur le serveur esclave. Le besoin en haute disponibilité n'est pas primordial. 

\subsection{Principe}

Le mode de réplication correspondant au besoin est \textit{maître/esclave}. On peut lire et écrire sur le maître et seulement lire sur l'esclave s'il est configuré en \textit{hot standby}. Ici, le serveur maître est \textit{citerne-8} et le serveur esclave est \textit{chappie}.

Les modifications de données sont enregistrées dans des journaux de transactions appelés \textbf{WAL (Write-Ahead Log) xlogs}. Ces WAL sont transférés à l'esclave qui les rejoue continuellement de sorte à se retrouver dans le même état que le maître. Il sera alors prêt à prendre la relève en cas d'indisponibilité du maître.

Grâce au principe de \textit{Streaming replication}, on n'attend plus que le fichier WAL (16 Mio) soit rempli mais il sera transmis sans délai du maître à l'esclave.

\subsection{Limitations et précautions}
Dans la configuration, comme on va conserver 256 xlogs à l'aide du paramètre \textbf{wal\_keep\_segments}, il faut prévoir assez d'espace disque disponible.

La réplication entre deux serveurs de versions différentes de postgresql est impossible.

\section{Mise à jour du serveur (version 9.3) en version 9.4 dans \textit{citerne-8:}}

On installe la dernière version de postgresql, on liste les clusters qui tournent et on supprime le cluster 9.4 existant:

\begin{lstlisting}
root@citerne-8:~# apt-get install postgresql-9.4
root@citerne-8:~# pg_lsclusters
root@citerne-8:~# pg_dropcluster --stop 9.4 main
\end{lstlisting}

Mise à jour du cluster :

\begin{lstlisting}
root@citerne-8:~# pg_upgradecluster 9.3 main 
\end{lstlisting}

Liste des clusters et visualisation de leur activité : 
\begin{lstlisting}
root@citerne-8:~# pg_lsclusters
\end{lstlisting}

Suppression de l'ancien cluster :
\begin{lstlisting}
root@citerne-8:~# pg_dropcluster --stop 9.3 main
\end{lstlisting}

Modification du port du cluster 9.4 dans le fichier \textit{/etc/postgresql/9.4/main/postgresql.conf} :
\begin{lstlisting}
port = 5432
\end{lstlisting}


\section{Installation de postgreSQL sur \textit{chappie} et mise en place des clés ssh}
\begin{lstlisting}
root@chappie:~# apt-get install postgresql-9.4
root@chappie:~# su - postgres
postgres@chappie:~$ mkdir /var/lib/postgresql/.ssh/
postgres@chappie:~$ ssh-keygen
\end{lstlisting}

Pour la connexion ssh entre les deux serveurs, il faut mettre la clé de l'utilisateur postgres contenue dans le fichier \textbf{id\_rsa.pub} sur \textit{chappie} dans le fichier \textbf{authorized\_keys} de \textit{citerne-8} et inversement.

\section{Mise en place de la réplication}

\subsection{ Maître}

Création de l'utilisateur posgresql chargé de la réplication :
\begin{lstlisting}
root@citerne-8:~# su - postgres
postgres:~$ psql -c "CREATE USER rep REPLICATION LOGIN ENCRYPTED PASSWORD 'desperados';"
\end{lstlisting}

Dans le fichier \textbf{pg\_hba.conf} (\textit{/etc/postgresql/9.4/main/}) ajoutez :
\begin{lstlisting}
host    replication     rep     10.33.192.31/32   md5
\end{lstlisting}

Pour le paramètre \textbf{wal\_keep\_segments}, on lui donne une valeur assez grande pour éviter d'accumuler un retard trop important entre les deux serveurs en cas d'indisponibilité de l'esclave.

Dans le fichier \textbf{postgresql.conf} ajoutez ces lignes\footnote{Attention: Si vous faites un copier-coller, les apostrophes ne sont pas des apostrophes droites donc il faudra les modifier} :

\begin{lstlisting}
listen_address = 'localhost,10.33.192.36' 
wal_level = hot_standby 
max_wal_senders = 3 
max_wal_size = 436MB 
wal_keep_segments = 256 
\end{lstlisting}

Redémarrez ensuite le service postgresql.

\subsection{Esclave}
\label{esclave}
Arrêtez le service postgresql, puis ajoutez ces lignes dans le fichier \textbf{postgresql.conf} :
\begin{lstlisting}
wal_level = hot_standby
max_wal_senders = 3
max_wal_size = 384MB
wal_keep_segments = 256
hot_standby = on
max_locks_per_transaction = 128
\end{lstlisting}

Modifiez le fichier \textbf{pg\_hba.conf} :
\begin{lstlisting}
host    replication     rep     10.33.192.36/32 md5 
\end{lstlisting}

Effectuez la sauvegarde complète des bases du serveur maître (depuis l'esclave, toujours) avec l'utilisateur postgres :
\begin{lstlisting}
pg_dropcluster 9.5 main
pg_basebackup -h 10.33.192.36 -D /var/lib/postgresql/9.5/main -U rep -v -P --xlog
\end{lstlisting}

L'option --xlog est ajoutée pour garder les derniers journaux de transactions.

Créez le fichier \textbf{recovery.conf} dans \textit{/var/lib/postgresql/9.5/main/} pour configurer la restauration continue.

La restauration en continu s'active à l'aide du paramètre \textit{standby\_mode}. Pour se connecter au maître et récupérer les WAL, on définit les informations nécessaires dans le paramètre \textit{primary\_conninfo}. 

Le paramètre \textit{trigger\_file} indique si la restauration doit être interrompue (si le fichier indiqué est présent, le processus est arrêté).
\begin{lstlisting}
standby_mode = on 
primary_conninfo = 'host=10.33.192.36 port=5432 user=rep password=desperados' 
trigger_file = '/var/lib/postgresql/9.4/postgresql.trigger' 
\end{lstlisting}

Pour finir, démarrez le service postgresql.

\section{Informations de monitoring}

Le fichier de logs \textbf{postgresql-9.4-main.log} se trouve dans le répertoire \textit{/var/log/postgresql/}

Pour savoir où en est la réplication du côté du maître :
\begin{lstlisting}
sudo -u postgres psql -x -c "select * from pg_stat_replication;"
\end{lstlisting}

Pour savoir à quand remonte la dernière synchronisation du côté de l'esclave :
\begin{lstlisting}
sudo -u postgres psql -x -c "SELECT now() - pg_last_xact_replay_timestamp() AS time_lag;"
\end{lstlisting}

Pour voir le numéro du snapshot actuel :
\begin{lstlisting}
sudo -u postgres psql -x -c "SELECT txid_current_snapshot();"
\end{lstlisting}


\section{Pour tester le failover ou gérer un interruption}

Le serveur maître est indisponible.

Il faut arrêter la restauration continue sur l'esclave pour qu'il devienne le maître, en créant le fichier \textit{trigger}. Les bases vont alors passer en mode read/write et le fichier \textit{recovery.conf} sera renommé \textit{recovery.done}.
\begin{lstlisting}
sudo touch /var/lib/postgresql/9.4/postgresql.trigger
\end{lstlisting}

Lorsque le maître sera de retour, la réplication ne fonctionnera plus. Vous devrez restaurer les données provenant du serveur esclave dans le serveur maître, puis relancer la réplication, en recréant le fichier \textit{recovery.conf}, comme décrit dans la section \ref{esclave} \textit{\nameref{esclave}}.

