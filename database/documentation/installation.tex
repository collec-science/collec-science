\chapter{Installer le logiciel}

\section{Consultez la documentation du framework !}

Le logiciel a été conçu à partir du framework \textit{Prototypephp}. La documentation associée récapitule l'ensemble des informations nécessaires pour réaliser l'installation générale (configuration du serveur, définition des droits d'accès, etc.).

De nombreuses reprises figurent ici, mais il n'est pas inutile de se référer au document d'origine. Seuls les exemples dédiés au logiciel sont présentés, mais d'autres informations utiles pourront y être piochées.

\section{Configuration du serveur}
La configuration est donnée pour un serveur Linux fonctionnant avec Ubuntu 16.04 LTS Server. Elle peut bien sûr être adaptée à d'autres distributions Linux. Par contre, rien n'a été prévu pour faire fonctionner l'application directement dans une plate-forme windows, même si, en théorie, cela devrait être possible.

\subsection{Pré-requis}

\subsection{Configuration de l'hôte virtuel et de SSL}

\subsection{Configuration du dossier d'installation}

\subsubsection{Mécanisme pour faire cohabiter plusieurs instances avec le même code}

\subsubsection{Droits à attribuer au serveur web}